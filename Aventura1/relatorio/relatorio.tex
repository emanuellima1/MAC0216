\documentclass{article}
\usepackage[utf8]{inputenc}
\usepackage{spverbatim}


\title{Relatório - Parte I da Aventura}
\date{26/09/2018}
\author{Emanuel Lima e João Seckler} 


\begin{document}
\maketitle

Para usar o programa teste.c, execute:

\begin{spverbatim}
$ gcc -Wall -o teste elemento.c lista.c tabela.c teste.c
$ ./teste.c
\end{spverbatim}
  
\medskip

O programa de teste deve imprimir uma mensagem na saída padrão se houver algum  erro.

Optamos por definir dois tipos para elemento: "elemento" e "Elemento". O segundo é um ponteiro para o primeiro. A desvantagem é que precisamos criar duas funções "elemento\_cria" e "elemento\_destroi", mas em compensação nos parece que o restante do código, nos módulos de lista e tabelas, fica mais limpo e legível. Fazemos isso inspirados na implementação de um módulo de pilhas feita em MAC0121 - Algoritmos e Estruturas de Dados.

Com essa diferença, algumas funções têm sua declaração alterada:

Lista insere(Lista l, Elemento *val)
torna-se Lista
insere(Lista l, Elemento val);

Elemento *busca(Lista l, char *n)
torna-se
Elemento busca(Lista l, char *n); e

Elemento *retira(Lista l, Elemento *val)
torna-se
Elemento retira(Lista l, Elemento val)

Além disso, decidimos não seguir a sugestão do enunciado para a implementação da lista ligada à risca. Ele sugeria que o tipo lista fosse definido como:

typedef struct {
  Elo * cabec ;
} Lista ;

No entanto, optamos por definir o tipo Lista simplesmente como um ponteiro para o tipo Elo. Assim, a cabeça de toda lista é criada pela função cria\_lista, e é ela mesma um "Elo" cujo "val" não aponta para nenhum lugar relevante. Se é verdade que essa implementação gasta o espaço desse "val", que não é usado, também é verdade que evitamos novas definições, que pareciam tornar o código menos claro. O benefício apresentado pelo enunciado, qual seja, "o endereço da lista não muda com inserções e deleções", continua valendo, pela definição de "Lista". Essa estratégia é inspirada nas página sobre lista encadeada do Prof. Paulo Feofiloff (https://www.ime.usp.br/~pf/algoritmos/aulas/lista.html).

Na implementação da função tabela\_insere(Tabela T, char *n, Elemento val), optamos por devolver um código de erro se o valor val associado à chave n já estiver armazenado no mesmo índice de T. Esse evento pode ter dois significados: (i) que o usuário tentou associar uma mesma chave a um mesmo valor mais de uma vez ou (ii) que o usuário associou duas chaves a um mesmo valor, e essas chaves colidem na função hash.

Referência para a função de hash: http://www.cse.yorku.ca/~oz/hash.html 

\end{document}